% !TeX root = ../main.tex
% Add the above to each chapter to make compiling the PDF easier in some editors.

\chapter{Prior Work}\label{chapter:prior_work}

A great deal of research has been done in the area of Empirical Software Engineering over the last 15 years. Many topics have been examined already; topics like: developers’ working rhythm, experience, focus, and productivity; use of Q\&A websites like StackOverflow; ownership of code; and their influence on code quality\parencite{Sliwerski,Izquierdo-Cortazar,tsunoda,qiu,Eyolfson,Mo,Vasilescu:2016:SLM}. Numerous research papers studied code quality. In “When Do Changes Induce Fixes?,” Sliwerski et al. described an automated method for finding fix-inducing changes. The method was since known as the SZZ algorithms\parencite{Sliwerski}. Sliwerski et al. found that larger changes and modifications made on Fridays tend to contain more defects. \par

Izquierdo-Cortazar et al. studied the relationship between a developer’s experience and bug introduction ratio and they found no correlation\parencite{Izquierdo-Cortazar}; while, in a similar research, Tsunoda et al. found that more experienced developers introduce fewer bugs if the work complexity is kept on the same level\parencite{tsunoda}. Qiu et al. performed an empirical study of the developer quality metrics and discovered that the quality increases with higher contributions and with software evolution\parencite{qiu}. In “Do time of Day and Developer Experience Affect Commit Bugginess?” Eyolfson et al. studied the relationship of the social characteristics of  commits and their "bugginess." They found that late night changes introduce more defects and that developers who commit to a project on a daily basis introduce fewer bugs\parencite{Eyolfson}. \par

Since GitHub became a popular open source version control system, researchers have used it to understand developers’ practices and their influence on programming quality. Mo et al. analyzed programmer’s commiting behaviors and social network in GitHub to find experts in selected programming languages\parencite{Mo}. In “The Sky is Not the Limit,” Vasilescu et al. analyzed the influence of project-level focus on developers’ productivity. They found that limited switching between projects is beneficial\parencite{Vasilescu:2016:SLM}. In another research, Vasilescu et al. analyzed the relationship between activity on StackOverflow, a Q\&A website for developers, and GitHub. They found that more active developers ask less and answer more questions, and that they distribute their code in more uniform ways\parencite{Vasilescu:2013:SO}. \par

The paper most similar to my research is “Dual Ecological Measure of Focus in Software Development” by Posnett et al. In their paper, Posnett et al. looked at the question of whether focused developers produce better-quality code and whether modules that receive narrower focus are of better quality\parencite{posnett}. In their research, the authors used module-based focus measure and found that more focused developers introduce fewer defects, while the modules that receive narrow focus from developers are more likely to contain bugs. In my research, I attempted to find an answer to the question from the opposite side. When one looks at a set of focused developers and a set of \textit{good developers}, the two sets may overlap, but they may not be the same. By analyzing focus of \textit{good developers} instead of the quality of focused developers, I was able to answer the question of  whether \textit{good developers} are more focused. Also, in differentiation from Posnett et al., this paper used extension-based focus, which is described in detail in the Background and Methods sections. Additionally, in my research, I went further to analyze focus over different time periods and to find out how developers interact with each file extension.



