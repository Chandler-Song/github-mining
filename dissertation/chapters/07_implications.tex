% !TeX root = ../main.tex
% Add the above to each chapter to make compiling the PDF easier in some editors.

\chapter{Implications to the Field}\label{chapter:implications}

With my research, I seek to contribute to the field of Empirical Software Engineering. It is important to understand the features and practices of \textit{good developers}. Such insights constitute valuable feedback for the community and paths of improvement for the beginner programmers. This knowledge can lead to improving programming and analytics tools like IDEs, or to validating software processes and languages. \par

In this paper, I studied the question of whether \textit{good developers} are more focused. By examining  the commits, I was able to analyze first the programmers’ quality, then their focus over different periods of time. In addition, I inspected the extension-specific focus. The research findings include:
\begin{itemize}
\item \textit{Good developers} tend to be more focused over shorter time periods but they do not stick to one technology throughout their career;
\item A majority of developers schedule their work in a way that each day depends on the previous days;
\item Overall, \textit{good developers} display higher focus per extension;
\item Among the 10 most popular extensions, Ruby files are those which receive lower amount of  focus from both groups;
\item \textit{Good developers} focus more on all the most popular extensions except .rb;
\item \textit{Good developers} are most focused on the following extensions:  .bat, .map, .rc, .sql, .policy; while \textit{bad developers} do not focus on any extension.
\end{itemize}

Data from the empirical analysis of developers’ practices can empower development teams to improve their working habits. Tools like IDEs or version control systems should be able to provide programmers with the focus statistics that can allow developers to distribute their tasks in the most optimal way. Such data can encourage developers to double-check their code if their focus for the day was significantly lower. It can also warn both programmers and managers ahead of time. If the focus level for the day is low, developers can spend the rest of the day focusing more on one technology instead of spreading themselves thin within the project. The focus statistics can also help programmers reflect on their habits and improve over time. It can be also used as an aid to plan daily tasks. \par

Additionally, information about developers’ focus and quality provides useful feedback for project managers. The data may be used to plan maintenance efforts and allocate developers resources within the project. The allocation can be done based on task complexity and the technology involved. Each developer can be assigned to the technology they focused on the most in the past, and to the tasks with proper difficulty based on the past code quality.\par

