\chapter{\abstractname}

The modern-day Software Engineering field requires diverse skills from programmers. With the growth of Open Source Software (OSS), most contributors work on many different technologies throughout their career, often working on multiple technologies each day. However, switching between technologies and coding styles may adversely affect developers’ focus. It is important to understand what makes programmers thrive and what can affect their skills negatively. In order to improve their abilities, developers often look up to other established developers. Experience and productivity are some of the traits believed to characterize \textit{good developers} to whom other programmers look up. But what about their focus? To answer the question of whether \textit{good developers} are more focused, I analyzed focus of \textit{good developers} based on the number of extensions they work on within a specified time period. \par

This paper used a method of evaluating buggy lines in developers’ contributions to divide a group of authors into \textit{good} and \textit{bad}. The results are presented in the form of a high-level focus comparison between two groups. I have found that \textit{good developers} are more focused daily and even yearly, but they diversify the technologies they use over their careers. I also investigated whether developers interact differently with selected technologies. \textit{Good developers} focus more than others on nine out of ten most popular extensions. All developers have the same focus level for .rb. 
The findings can help developers reflect on their programming practices, improve their code quality, plan projects better. An additional research could extend this study with a more fine-grained definition of developers’ focus that takes into account temporal switching between technologies. Also, other characteristics that affect the quality of developers could be investigated in order to present a full picture of what makes a \textit{good developer}.


\makeatletter
\ifthenelse{\pdf@strcmp{\languagename}{english}=0}
{\renewcommand{\abstractname}{Abstract}}
\makeatother

